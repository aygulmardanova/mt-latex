/**
* Calculates LCSS for two input trajectories
*
* @param t1			first trajectory
* @param t2			second trajectory
* @param (*$\delta$*)			 (*$\delta$*) parameter: how far we can look in time to match a given point from one T to a point in another T
* @param (*$\varepsilon$*)			 (*$\varepsilon$*) parameter: the size of proximity in which to look for matches
* @return 			LCSS for t1 and t2
*/
private Double calcLCSS(Trajectory t1, Trajectory t2, Double (*$\delta$*), Double (*$\varepsilon_x$*), Double (*$\varepsilon_y$*)) {
	int m = t1.length();
	int n = t2.length();
	
	if (m == 0 || n == 0) {
		return 0.0;
	}

//        calculate adaptive (*$\varepsilon$*) for X and Y
	if (IS_ADAPTIVE) {
		TrajectoryPoint tp1 = t1.getKeyPoints().get(m - 1);
		TrajectoryPoint tp2 = t2.getKeyPoints().get(n - 1);
		epsilonX = getEpsilonX(tp1, tp2, LinkageMethod.AVERAGE);
		epsilonY = getEpsilonY(tp1, tp2, LinkageMethod.AVERAGE);
	}
	
//      check last trajectory point (of each trajectory-part recursively)
//      according to [8]: delta and epsilon as thresholds for X- and Y-axes respectively
//      Then the abscissa difference and ordinate difference are less than thresholds (they are relatively close to each other)
//      they are considered similar and LCSS distance is increased by 1
	if (abs(t1.get(m - 1).getX() - t2.get(n - 1).getX()) < (*$\varepsilon_x$*)
			&& abs(t1.get(m - 1).getY() - t2.get(n - 1).getY()) < (*$\varepsilon_y$*)
			&& abs(m - n) <= (*$\delta$*)) {
		return 1 + calcLCSS(head(t1), head(t2), (*$\delta$*), (*$\varepsilon$*));
	} else {
		return max(
			calcLCSS(head(t1), t2, (*$\delta$*), (*$\varepsilon_x$*), Double (*$\varepsilon_y$*)), 
			calcLCSS(t1, head(t2), (*$\delta$*), (*$\varepsilon_x$*), Double (*$\varepsilon_y$*)));
	}
}

/**
* Calculates shortened trajectory by excluding last trajectory point
*
* @param t trajectory
* @return trajectory without last trajectory point
*/
private Trajectory head(Trajectory t) {
	Trajectory tClone = t.clone();
	tClone.getTrajectoryPoints().remove(tClone.length() - 1);
	return tClone;
}
