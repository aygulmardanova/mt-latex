%%% Preamble with settings %%%
\usepackage[english]{babel}
\usepackage[utf8]{inputenc}
\usepackage{fancyhdr}
\usepackage{setspace}
\onehalfspacing

%% ToDo notes %%
\usepackage{todonotes}

%% Bibliography %%%
\usepackage[square,numbers]{natbib}
\bibliographystyle{unsrt}

\usepackage[nottoc]{tocbibind} % Includes "Bibliography" in contents

\usepackage[acronym]{glossaries}
\makeglossaries

\usepackage{fontspec} % XeTeX
\usepackage{xunicode} % Unicode для XeTeX
\usepackage{xltxtra}  % Верхние и нижние индексы
\usepackage{indentfirst} % Indented line after header

\usepackage{listings} % Listings format settings
\lstset{
    basicstyle=\small\ttfamily, % Размер и тип шрифта
    breaklines=true, % Перенос строк
    tabsize=2, % Размер табуляции
   frame=single,               % Рамка
   literate={--}{{-{}-}}2,     % Корректно отображать двойной дефис
   literate={---}{{-{}-{}-}}3  % Корректно отображать тройной дефис
}

% Шрифты, xelatex
\defaultfontfeatures{Ligatures=TeX}
\setmainfont{Times New Roman}
\setmonofont{FreeMono}

%% Graphics & Illustrations %%%%%%%%%%%%%%%%%%%%%%%%%
\usepackage{graphicx}			% Images and additions inserting (to load graphics)
\usepackage{wrapfig}			% integration of graphics with wraping text
\usepackage{subfig}				% integration of multiple objects within a float
\usepackage{pdfpages}			% PDF inserting, binds a graphic or page (.pdf or .jpg) in the document
\usepackage{float}

% Format of labels for figures and tables
\usepackage{chngcntr}

% Reset the counter for figures, tables in each Chapter
\counterwithin{figure}{section}
\counterwithin{table}{section}

% Format of labels for theorems and definitions
\usepackage{amssymb,amsfonts,amsmath, amsthm} % Maths
\numberwithin{equation}{section} % Formules numbering chapter.number

\theoremstyle{definition}
\newtheorem{definition}{Definition}

\usepackage[numbers]{natbib}
\usepackage[yyyymmdd,hhmmss]{datetime}
\usepackage[ruled,vlined]{algorithm2e}

%% HyperRef %%%%%%%%%%%%%%%%%%%%%%%%%%%%%%
% http://www.tug.org/applications/hyperref/manual.html
\usepackage{hyperref}
\hypersetup{
    colorlinks, urlcolor={black}, % All links are black and clickable
    linkcolor={black}, citecolor={black}, filecolor={black},
    pdfauthor={Aigul Mardanova},
    pdftitle={Identification of Trajectory Anomalies in Uncertain Spatiotemporal Data}
}

\sloppy             % Избавляемся от переполнений
\hyphenpenalty=1000 % Частота переносов
\clubpenalty=10000  % Запрещаем разрыв страницы после первой строки абзаца
\widowpenalty=10000 % Запрещаем разрыв страницы после последней строки абзаца

% Списки
\usepackage{enumitem}
\setlist[enumerate,itemize]{leftmargin=12.7mm} % Отступы в списках

% Page number is bottom-right
\usepackage{fancyhdr}
\pagestyle{fancy}
\fancyhf{}
\rhead{\rightmark}
\fancyfoot[R]{\thepage}
\renewcommand{\headrulewidth}{0.4pt}
\renewcommand{\footrulewidth}{0.4pt}

\setcounter{page}{4} % Начало нумерации страниц